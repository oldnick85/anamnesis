\documentclass[12pt,a4paper,oneside]{article}
\usepackage[utf8]{inputenc}
\usepackage[russian]{babel}
\usepackage[OT1]{fontenc}
\usepackage{amsmath}
\usepackage{amsfonts}
\usepackage{amssymb}
\title{Основание программного обеспечения \textbf{Anamnesis}}
\begin{document}
\maketitle

\tableofcontents
\newpage

%%%%%%%%%%%%%%%%%%%%%%%%%%%%%%%%%%%%%%%%%%%%%%%%%%%%%%%%%%%%%%%%

\section{Теоретическое обоснование}

\subsection{Основные определения и терминология}

\textbf{Паролем} мы будем называть конечную последовательность (кортеж) символов $\alpha=(\alpha_{i_0},\ldots,\alpha_{i_{n-1}})$ произвольного алфавита $\mathbb{A}=\{\alpha_0, \ldots, \alpha_{b-1}\}$, где $n$ --- длина пароля, $b$ --- мощность алфавита.

\textbf{Цифровым паролем (ЦП)} мы будем называть представление некоторого натурального числа $a$ в позиционной системе счисления по основанию $b$: $a=\sum^{n-1}_{k=0}a_kb^k$, где $a_i$ --- цифры числа в этой системе счисления.

Если символы алфавита $\mathbb{A}$ пронумерованы числами $0\ldots b-1$, т.\,е. задана биекция $\phi:\mathbb{A}\rightarrow\{0\ldots b-1\}$, то любому паролю взаимно-однозначно соответствует некоторый цифровой пароль как значение
$$
\Phi(\alpha)=\sum^{n-1}_{k=0}\phi(\alpha_k)b^k.
$$
В дальнейшем будем называть ЦП паролем в тех случаях, когда такая биекция очевидна или подразумевается.

\textbf{Сейфом пароля} или просто \textbf{сейфом} будем называть натуральное число $A$, заданной в смешанной системе счисления следующим образом:
$$
A=\sum^{N-1}_{k=0}A_kB_k.
$$
Здесь $A_k$ --- цифры числа $A$, а $B_k=\prod^{k}_{i=0}C_i$, где $C_0=1, C_1,\ldots,C_k$ --- некоторые натуральные числа.

\section{Виды ассоциативных частей}

\subsection{Комбинаторные значения}
Поскольку комбинаторные задачи приводят к достаточно большим значениям на относительно небольших множествах, они хорошо подходят для кодирования паролей.

\begin{itemize}
\item Упорядоченная выборка без повторений, размещение из $n$ элементов по $k$, обозначаемое $A^k_n$. Частный случай $A_n=A^n_n$ --- перестановка из $n$ элементов.
\item Неупорядоченная выборка без повторений, сочетание из $n$ элементов по $k$, обозначаемое $C^k_n$.
\item Упорядоченная выборка с повторениями, обозначаемое $\overline{A}^k_n$.
\end{itemize}

\subsection{Нумерация комбинаторных значений}

\subsubsection{Перестановка}
Общее количество перестановок равно $A_n=n!$.
Пусть выборка имеет вид $S_n=\left(i_1,i_2,\ldots,i_n\right)$, где $i_p$ --- попарно различные числа. Занумеруем множество перестановок $S(A_n)$ натуральными числами, т.\,е. определим биекцию $\vartheta: S(A_n)\leftrightarrow\left[0\ldots n!-1\right]$.

Определим преобразование $O_{S_n}(i_p)\mapsto \left| \left\{i < i_p | i \in S_n \right\} \right|$. Другими словами, $O_{S_n}$ ставит в соответствие элементу кортежа его порядковый номер по возрастанию, наименьшему элементу соответствует ноль.

Тогда можно определить нумерацию $\vartheta$ следующим образом:
$$
\vartheta(S_n) := O_{S_n}(i_1)\cdot(n-1)! + \vartheta(i_2,\ldots i_{n}).
$$
$$
\vartheta(\emptyset) := 0.
$$

Исходя из определения очевидно, что $\vartheta(i,j)=0$, если $i<j$ и $\vartheta(i,j)=1$, если $i>j$. Таким образом для $n=2$ свойства нумерации выполняются. Если предположить, что эти свойства выполняются для некоторого $n$, то можно доказать их выполнимость для $n+1$. Действительно, 
\begin{eqnarray*}
\vartheta(S_{n+1}) = O_{S_n}(i_1)\cdot n! + \vartheta(S_{n}) 
= \left[0;n\right]\cdot n! + \left[0;n!-1\right] =\\
= \left[0+0;n\cdot n! + n! - 1\right] 
= \left[0;(n+1)!-1\right].
\end{eqnarray*}

Примеры вычисления:
\begin{eqnarray*}
\vartheta(1,2,3,4) = 0\cdot3!+\vartheta(2,3,4) = 0\cdot3!+0\cdot2!+\vartheta(3,4)=\\
= 0\cdot3!+0\cdot2!+0\cdot1! = 0.
\end{eqnarray*}
\begin{eqnarray*}
\vartheta(2,4,1,3) = 1\cdot3!+\vartheta(4,1,3) = 1\cdot3!+2\cdot2!+\vartheta(1,3)=\\
= 1\cdot3!+2\cdot2!+0\cdot1! = 10.
\end{eqnarray*}
\begin{eqnarray*}
\vartheta(4,3,2,1) = 3\cdot3!+\vartheta(3,2,1) = 3\cdot3!+2\cdot2!+\vartheta(2,1)=\\
= 3\cdot3!+2\cdot2!+1\cdot1! = 23 = 24!-1.
\end{eqnarray*}

\subsubsection{Упорядоченная выборка без повторений}
Общее количество упорядоченных выборок без повторений равно $A^k_n=\frac{n!}{(n-k)!}$.
Пусть выборка имеет вид $S^k_n=\left(i_1,i_2,\ldots,i_k\right)$, где $i_p$ --- попарно различные натуральные числа из множества $S_n=\left\{i_1,i_2,\ldots,i_n\right\}$. Занумеруем множество перестановок $S(A^k_n)$ натуральными числами, т.\,е. определим биекцию $\vartheta: S(A^k_n)\leftrightarrow\left[0\ldots \frac{n!}{(n-k)!}-1\right]$.

Определим преобразование $O_{S_n}(i_p)\mapsto \left| \left\{i < i_p | i \in S_n \right\} \right|$. Другими словами, $O_{S_n}$ ставит в соответствие элементу кортежа его порядковый номер по возрастанию в исходном множестве, наименьшему элементу соответствует ноль.

Тогда можно определить нумерацию $\vartheta$ следующим образом:
$$
\vartheta(S^k_n) := O_{S^k_n}(i_1)\cdot\frac{(n-1)!}{(n-k)!} + \vartheta(i_2,\ldots i_{n}).
$$
$$
\vartheta(\emptyset) := 0.
$$

Примеры вычисления $A^3_5$ для исходного множества $\left\{1,2,3,4,5\right\}$:
\begin{eqnarray*}
\vartheta(1,2,3) = 0\cdot\frac{4!}{2!}+\vartheta(2,3) = 0\cdot\frac{4!}{2!}+0\cdot\frac{3!}{2!}+\vartheta(3) =\\
= 0\cdot\frac{4!}{2!}+0\cdot\frac{3!}{2!}+0\cdot\frac{2!}{2!} = 0.
\end{eqnarray*}
\begin{eqnarray*}
\vartheta(1,2,4) = 0\cdot\frac{4!}{2!}+\vartheta(2,4) = 0\cdot\frac{4!}{2!}+0\cdot\frac{3!}{2!}+\vartheta(4) =\\
= 0\cdot\frac{4!}{2!}+0\cdot\frac{3!}{2!}+1\cdot\frac{2!}{2!} = 1.
\end{eqnarray*}
\begin{eqnarray*}
\vartheta(5,4,3) = 4\cdot\frac{4!}{2!}+\vartheta(4,3) = 4\cdot\frac{4!}{2!}+3\cdot\frac{3!}{2!}+\vartheta(3) =\\
= 4\cdot\frac{4!}{2!}+3\cdot\frac{3!}{2!}+2\cdot\frac{2!}{2!} = 59 = \frac{5!}{2!}-1.
\end{eqnarray*}

\end{document}
